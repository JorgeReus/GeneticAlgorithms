\documentclass{article}
\usepackage{style}
\begin{document}
\maketitle
\tableofcontents
\section{Introducción}
El algoritmo genético enfatiza la importancia de la cruza y la mutación, al igual que la selección probabilistica.
En esta práctica se implementó el algoritmo genético más simple:
\begin{itemize}
	\item Generar aleatoriamente una población inicial.
	\item Calcular la aptitud de cada individuo.
	\item Seleccionar probabilisticamente con base en la aptitud.
	\item Aplicar cruza y mutación para generar la siguiente población.
	\item Ciclar hasta que las condiciones finales se cumplan.
\end{itemize}
Esta técnica fue propuesta por DeJong [137], y ha sido el método más comúnmente
usado desde los orígenes de los algoritmos genéticos. El algoritmo es simple, pero
ineficiente (su complejidad es O(n 2 ). Asimismo, presenta el problema de que el
individuo menos apto puede ser seleccionado más de una vez.
\newpage
\section{Contenido}
\subsection{Algoritmo con 5 generaciones}
\begin{figure}[h!]
	\centering
	\includegraphics[scale=.3]{5gen}
\end{figure}
\newpage
\subsection{Algoritmo con 10 generaciones}
\begin{figure}[h!]
	\centering
	\includegraphics[scale=.3]{10gen}
\end{figure}
\subsection{Algoritmo con 15 generaciones}
\begin{figure}[h!]
	\centering
	\includegraphics[scale=.3]{15gen}
\end{figure}
\section{Conclusión}
Este algoritmo es el algoritmo genético más simple, pero el más popular. En esta práctica se usó la selección por ruleta, la cual puede seleccionar al individuo menos apto. Los individuos más aptos son los que tiene más probabilidad de ser seleccioandos.
\end{document}